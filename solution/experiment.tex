\documentclass[11pt,a4paper]{article}

\usepackage{fontspec}
\setmainfont{Times New Roman}
\setsansfont{Arial}
\setmonofont{Courier New}

\usepackage[BoldFont,SlantFont,CJKchecksingle,CJKnumber]{xeCJK}
\setCJKmainfont[BoldFont={Adobe Heiti Std},ItalicFont={Adobe Kaiti Std}]{Adobe Song Std}
\setCJKsansfont{Adobe Heiti Std}
\setCJKmonofont{Adobe Fangsong Std}
\punctstyle{hangmobanjiao}

\defaultfontfeatures{Mapping=tex-text}
\usepackage{xunicode}
\usepackage{xltxtra}

\XeTeXlinebreaklocale "zh"
\XeTeXlinebreakskip = 0pt plus 1pt minus 0.1pt

%\usepackage{indentfirst}
\makeatletter
\let\@afterindentfalse\@afterindenttrue
\@afterindenttrue
\makeatother
\setlength{\parindent}{2em}

\linespread{1.5}

\usepackage{xcolor}

\usepackage[xetex,bookmarksnumbered=true,bookmarksopen=true,pdfborder=1,breaklinks,colorlinks,linkcolor=blue,urlcolor=blue,citecolor=blue]{hyperref}

\usepackage[top=1.2in,bottom=1.2in,left=1.2in,right=1in]{geometry}

%\usepackage[pagestyles]{titlesec}
\usepackage{titlesec}
%\titleformat{\section}{\centering\Large\bfseries}{\S\,\thesection}{1em}{}
%\titleformat{\subsection}{\large\bfseries}{\S\,\thesubsection}{1em}{}

\usepackage{graphicx}
\graphicspath{{figures/}}

\renewcommand{\today}{\number\year 年 \number\month 月 \number\day 日}
\renewcommand{\contentsname}{目录}
\renewcommand{\listfigurename}{插图目录}
\renewcommand{\listtablename}{表格目录}
\renewcommand{\figurename}{图}
\renewcommand{\tablename}{表}
%\renewcommand{\bibname}{参考文献}
\renewcommand{\appendixname}{附录}
\renewcommand{\indexname}{索引}
\renewcommand{\abstractname}{摘要}
\renewcommand{\refname}{参考文献}

%\renewcommand{\equationautorefname}{公式}
%\renewcommand{\footnoteautorefname}{脚注}
%\renewcommand{\itemautorefname}{项}
\renewcommand{\figureautorefname}{图}
%\renewcommand{\tableautorefname}{表}
%\renewcommand{\appendixautorefname}{附录}
%\renewcommand{\theoremautorefname}{定理}



\begin{document}

\title{生物信息学课程设计题目(2009级)}
\author{伊现富}
\date{\today}
\maketitle

%%%%%%%%%%%%%%%%%%%%%%%%
%数据库、软件、平台、数据集以purple表示
%程序分组、子程序以blue表示
%参数、选项以orange表示
%参数值、选项值以cyan表示
%%%%%%%%%%%%%%%%%%%%%%%%

%EMBOSS
\begin{description}
	\item[名称:] 基于EMBOSS平台对***基因进行序列分析
	\item[总学时:] 36
	\item[学分:] 2
	\item[对象:] 生物医学工程系生物技术与生物信息专业本科生
	\item[指导教师:] 伊现富
	\item[课程设计目的:] 在学习分子生物学、生物计算技术、生物网络数据库、生物信息学课程的基础上,培养学生实际进行核酸序列和蛋白质序列分析的能力。此题目旨在帮助学生熟悉强大的EMBOSS生物信息学分析平台,并使用此平台进行基因序列与蛋白质序列的常规分析。
	\item[实验内容] \
		\begin{enumerate}
			\item 从NCBI Gene数据库中获取自己感兴趣的基因的DNA序列。
			\item 基因的核苷酸序列分析。
				\begin{enumerate}
					\item 统计该基因DNA序列的基本信息,如:长度、单核苷酸与二核苷酸的数目及频率、GC含量,等。
					\item 找出该基因DNA序列中的CpG岛。
					\item 提取该基因的编码区序列。
					\item 分析编码区序列的密码子使用频率。
					\item 寻找该基因序列中的ORF,与真实的编码区进行比较。
				\end{enumerate}
			\item 将编码区核苷酸序列翻译成氨基酸序列。
			\item 蛋白质的氨基酸序列分析。
				\begin{enumerate}
					\item 对翻译得到的蛋白质进行基本的理化性质分析。
					\item 寻找蛋白质中的功能与结构基序。
					\item 预测蛋白质的二级结构。
				\end{enumerate}
			\item 利用所有结果,分析该基因的生物学功能与特性。
		\end{enumerate}
	\item[技术指标] \
		\begin{enumerate}
			\item 下载基因的GenBank格式的序列文件。
			\item 利用GenBank文件中的注释信息,解读基因的基本信息(如基因全名、长度,所属物种,染色体定位,外显子、内含子等特征的位置,等)。
			\item 基因的核苷酸序列分析结果。
				\begin{enumerate}
					\item 基因DNA序列的单核苷酸、二核苷酸频率,GC含量等统计结果。
					\item CpG岛的预测结果及说明。
					\item 基因编码区序列的GenBank格式文件。
					\item 编码区的密码子使用频率与偏性分析。
					\item 基因ORF预测结果及其分析。
				\end{enumerate}
			\item 基因编码区翻译后的氨基酸序列文件。
			\item 蛋白质的氨基酸序列分析结果,即:蛋白质基本理化性质、基序及二级结构的分析结果。
			\item 整合上述数据,分析基因的生物学功能与特性。
			\item 课程设计报告书:(1)名称;(2)目的和任务;(3)实验步骤;(4)实验结果;(5)实验结果分析和讨论。
			\item 课程设计答辩的PPT文件(答辩时间5分钟)。
		\end{enumerate}
	\item[实验步骤] \ 
		\begin{enumerate}
			\item 打开 \textcolor{purple}{NCBI Gene} 数据库,在搜索栏中输入基因名\textit{G};在搜索结果中点击某个物种\textit{S}的链接,进入\textit{S}物种\textit{G}基因的信息界面;下拉至 \textcolor{orange}{NCBI Reference Sequences (RefSeq)} 项目,找到紧邻其下的 \textcolor{orange}{Genomic} 子项目,点击其中的 \textcolor{orange}{GenBank};在新页面的右侧点击 \textcolor{orange}{Send:},选择 \textcolor{orange}{File} 后,点击 \textcolor{orange}{Create File} 下载\textit{G}基因的GenBank格式的DNA序列。
			\item 对基因的DNA序列进行分析。打开 \textcolor{purple}{EMBOSS explorer} 界面。 
				\begin{enumerate}
					\item 组成成分分析。找到 \textcolor{blue}{NUCLEIC COMPOSITION} 分组。
						\begin{enumerate}
							\item 使用程序 \textcolor{blue}{compseq} 对DNA序列的基本组成成分进行分析。在 \textcolor{orange}{Input section} 项目中,使用 \textcolor{orange}{upload} 上传上一步从NCBI下载的GenBank格式的DNA序列;在 \textcolor{orange}{Required section} 项目中,把 \textcolor{orange}{word size} 分别设成 \textcolor{cyan}{1、2};其他参数默认,或者自行调整;最后,点击 \textcolor{orange}{Run compseq} 获得组成成分分析结果。
							\item 尝试使用程序 \textcolor{blue}{wordcount} 进行类似的组成成分分析。 尝试使用程序 \textcolor{blue}{chaos} 和 \textcolor{blue}{density} 将组成成分结果可视化。
						\end{enumerate}
					\item GC含量分析。找到 \textcolor{blue}{NUCLEIC CPG ISLANDS} 分组。使用程序 \textcolor{blue}{geecee}对DNA序列的GC含量进行分析。与前述类似,以上传文件的方式提交DNA序列,之后点击\textcolor{orange}{Run geecee} 得到DNA序列的GC含量。
					\item CpG岛分析。找到 \textcolor{blue}{NUCLEIC CPG ISLANDS} 分组。
						\begin{enumerate}
							\item 使用程序 \textcolor{blue}{newcpgreport} 寻找DNA序列中的CpG岛。上传DNA序列文件后,调整相应参数,点击 \textcolor{orange}{Run newcpgreport} 得到CpG岛的预测结果。
							\item 使用程序 \textcolor{blue}{cpgplot} 直观观察CpG岛的预测结果。上传DNA序列并调整参数,点击 \textcolor{orange}{Run cpgplot} 得到图形化的CpG岛预测结果。
							\item 尝试使用程序 \textcolor{blue}{cpgreport} 和 \textcolor{blue}{newcpgseek} 进行类似的CpG岛分析。
						\end{enumerate}
					\item 提取CDS序列。找到 \textcolor{blue}{FEATURE TABLES} 分组。
						\begin{enumerate}
							\item 使用程序 \textcolor{blue}{coderet} 提取基因中的CDS序列。上传DNA序列文件并调整输出格式为 \textcolor{cyan}{Genbank},点击 \textcolor{orange}{Run coderet} 得到该基因Genbank格式的CDS序列。将其保存至本地以备后用。
							\item 尝试使用程序 \textcolor{blue}{extractfeat} 提取CDS序列。尝试使用程序 \textcolor{blue}{showfeat} 将整个基因的特征可视化。
						\end{enumerate}
					\item 编码区密码子分析。找到 \textcolor{blue}{NUCLEIC CODON USAGE} 分组。
						\begin{enumerate}
							\item 使用程序 \textcolor{blue}{cusp} 计算CDS序列中密码子的使用频率。上传上一步保存的CDS序列文件,点击 \textcolor{orange}{Run cusp} 得到CDS序列的密码子使用频率。
							\item 尝试使用程序 \textcolor{blue}{syco} 对CDS序列中的同义密码子使用频率进行分析。
						\end{enumerate}
					\item ORF预测与分析。找到 \textcolor{blue}{NUCLEIC GENE FINDING} 分组。
						\begin{enumerate}
							\item 使用程序 \textcolor{blue}{getorf} 找到基因中的ORF。上传基因的DNA序列并调整参数,点击 \textcolor{orange}{Run getorf} 获得全部的ORF序列。
							\item 使用程序 \textcolor{blue}{plotorf} 和 \textcolor{blue}{showorf} 将基因中的ORF以图形方式直观显示出来。
							\item 将上述步骤的ORF序列结果与图形结果结合起来,并与基因实际的CDS进行比较。
						\end{enumerate}
				\end{enumerate}
			\item 翻译CDS序列。找到 \textcolor{blue}{NUCLEIC TRANSLATION} 分组。使用程序 \textcolor{blue}{transeq} 把CDS核苷酸序列翻译成氨基酸序列。上传CDS序列,调整参数与输出格式后,点击 \textcolor{orange}{Run transeq} 得到翻译后的氨基酸序列。将其保存以备后用。
			\item 对蛋白质的氨基酸序列进行分析。
				\begin{enumerate}
					\item 组成成分与理化性质分析。找到 \textcolor{blue}{PROTEIN COMPOSITION} 分组。
						\begin{enumerate}
							\item 使用程序 \textcolor{blue}{pepstats} 分析蛋白质的基本理化性质。上传翻译得到的氨基酸序列,调整参数后点击 \textcolor{orange}{Run pepstats} 得到蛋白质的基本理化性质信息。
							\item 使用程序 \textcolor{blue}{compseq} 查看蛋白质的基本组成成分。
							\item 使用程序 \textcolor{blue}{iep} 计算蛋白质的等电点。
							\item 尝试使用程序 \textcolor{blue}{charge} 获得每个氨基酸的带电量数据。尝试使用程序 \textcolor{blue}{octanol} 和 \textcolor{blue}{pepwindow} 对蛋白质的疏水性进行可视化。尝试使用程序 \textcolor{blue}{pepinfo} 对蛋白质氨基酸残基的极性进行分析。尝试使用 \textcolor{blue}{DISPLAY} 分组中的程序 \textcolor{blue}{pepnet} 和 \textcolor{blue}{pepwheel} 对蛋白质氨基酸残基的亲水性和疏水性进行可视化。
						\end{enumerate}
					\item 基序分析。找到 \textcolor{blue}{PROTEIN MOTIFS} 分组。
						\begin{enumerate}
							\item 使用程序 \textcolor{blue}{sigcleave} 寻找蛋白质中的信号切割位点。上传蛋白质序列后调整参数,点击 \textcolor{orange}{Run sigcleave} 找到蛋白质的信号切割为点。
							\item 使用程序 \textcolor{blue}{sigcleave} 寻找蛋白质中的信号切割位点。上传蛋白质序列后调整参数,点击 \textcolor{orange}{Run sigcleave} 找到蛋白质的信号切割为点。
							\item 使用程序 \textcolor{blue}{helixturnhelix} 寻找蛋白质上的核苷酸结合基序。上传蛋白质序列后调整参数,点击 \textcolor{orange}{Run helixturnhelix}寻找蛋白质上的核苷酸结合基序。
							\item 尝试使用此分组中的其他程序(如:\textcolor{blue}{antigenic})寻找更多的功能与结构基序。
						\end{enumerate}
					\item 结构分析。找到 \textcolor{blue}{PROTEIN 2D STRUCTURE} 分组。
						\begin{enumerate}
							\item 使用程序 \textcolor{blue}{garnier} 预测蛋白质的二级结构。上传蛋白质序列后调整参数,点击 \textcolor{orange}{Run garnier} 预测得到蛋白质的二级结构。
							\item 使用程序 \textcolor{blue}{tmap} 预测蛋白质中的跨膜片段。上传蛋白质序列后,点击 \textcolor{orange}{Run tmap} 预测得到蛋白质中的跨膜片段。
							\item 尝试使用此分组中的其他程序预测蛋白质的二级结构。
						\end{enumerate}
				\end{enumerate}
			\item 整合上述所有结果,分析基因\textit{G}的生物学功能、特性等。
		\end{enumerate}
	\item[学时分配] \ 
		\begin{description}
			\item[设计讲解:] 2学时
			\item[实验:] 18学时
			\item[总结和课程设计报告:] 12学时 
			\item[答辩:] 4学时
			\item[总共:] 36学时 
			\item[地点:] 教一楼205室生物信息学实验室 
		\end{description}
	\item[资源网站] \ 
		\begin{enumerate}
			\item EMBOSS官网:\href{http://emboss.sourceforge.net/}{http://emboss.sourceforge.net/}
			\item EMBOSS explorer:\href{http://emboss.bioinformatics.nl/}{http://emboss.bioinformatics.nl/};\href{http://bioinfo.nhri.org.tw/gui/}{http://bioinfo.nhri.org.tw/gui/};\href{http://genome.csdb.cn/emboss/}{http://genome.csdb.cn/emboss/}
			\item EMBOSS GUI:\href{http://anabench.bcm.umontreal.ca/html/EMBOSS/}{http://anabench.bcm.umontreal.ca/html/EMBOSS/};\href{http://bips.u-strasbg.fr/EMBOSS/}{http://bips.u-strasbg.fr/EMBOSS/}
			\item NCBI Gene:\href{http://www.ncbi.nlm.nih.gov/gene}{http://www.ncbi.nlm.nih.gov/gene}
		\end{enumerate}
\end{description}


\newpage
%Galaxy1
\begin{description}
	\item[名称:] 基于Galaxy平台分析***物种基因在基因组中的分布 
	\item[总学时:] 36
	\item[学分:] 2
	\item[对象:] 生物医学工程系生物技术与生物信息专业本科生
	\item[指导教师:] 伊现富
	\item[课程设计目的:] 在学习分子生物学、生物计算技术、生物网络数据库、生物信息学课程的基础上,培养学生分析大规模基因组数据的能力。此题目旨在帮助学生熟悉易用、强大且日渐流行的Galaxy生物信息学分析平台,并使用此平台处理基因组数据,分析基因在每条染色体上的分布情况。
	\item[实验内容] \
		\begin{enumerate}
			\item 选择合适的物种\textit{S}。
			\item 熟悉Galaxy界面。
			\item 获取该物种基因组范围的基因信息和每条染色体的长度。
			\item 计算每条染色体上的基因数目。
			\item 计算每条染色体上的基因密度。
			\item 绘制条形图将数据结果可视化。
			\item 结合数据表格和条形图分析结果。
		\end{enumerate}
	\item[技术指标] \
		\begin{enumerate}
			\item 基因组范围的基因数据。
			\item 基因组上每条染色体的长度。
			\item 每条染色体上的基因数目。
			\item 每条染色体上的基因密度。
			\item 基因数目与基因密度的条形图。
			\item 对所得结果的分析与理解。
			\item 课程设计报告书:(1)名称;(2)目的和任务;(3)实验步骤;(4)实验结果;(5)实验结果分析和讨论。
			\item 课程设计答辩的PPT文件(答辩时间5分钟)。
		\end{enumerate}
	\item[实验步骤] \ 
		\begin{enumerate}
			\item 选择完成基因组测序且注释比较完善的物种,如:人、小鼠,等。\footnote{此处以“人”为例详述步骤。}
			\item 打开 \textcolor{purple}{Galaxy Test} 主页,熟悉其界面布局。
			\item 获取基因与染色体信息数据。
				\begin{enumerate}
					\item 获取基因信息。
						\begin{enumerate}
							\item 获取数据。打开 \textcolor{purple}{Galaxy} 的 \textcolor{blue}{Get Data} 分组,点击 \textcolor{blue}{UCSC Main} 进入 \textcolor{purple}{UCSC Table} 界面。在 \textcolor{purple}{UCSC Table} 界面中,\textcolor{orange}{clade} 选择 \textcolor{cyan}{Mammal}、\textcolor{orange}{genome} 选择 \textcolor{cyan}{Human}、\textcolor{orange}{assembly} 选择 \textcolor{cyan}{Feb. 2009 (GRCh37/hg19)}、\textcolor{orange}{group} 选择 \textcolor{cyan}{Genes and Gene Prediction Tracks}、\textcolor{orange}{track} 选择 \textcolor{cyan}{RefSeq Genes}、\textcolor{orange}{table} 选择 \textcolor{cyan}{refGene}、\textcolor{orange}{region} 点选 \textcolor{cyan}{genome}、\textcolor{orange}{output format} 选择 \textcolor{cyan}{BED – browser extensible data}、\textcolor{orange}{Send output to} 勾选 \textcolor{cyan}{Galaxy}、\textcolor{orange}{file type returned} 点选 \textcolor{cyan}{plain text},之后点击 \textcolor{orange}{get output},新界面中的 \textcolor{orange}{Create one BED record per} 点选 \textcolor{cyan}{Whole Gene},最后点击 \textcolor{orange}{Send query to Galaxy} 即可将基因组中的基因信息载入到 \textcolor{purple}{Galaxy} 平台的工作空间中。
							\item 修改属性。为了便于区分工作空间中的不同数据,可以点击 \textcolor{purple}{1:UCSC Main on Human: refGene (genome)} 右侧的铅笔图标,修改数据集的属性,如:修改 \textcolor{orange}{Name} 为 \textcolor{cyan}{refGene},之后点击 \textcolor{orange}{Save} 更新其属性。
						\end{enumerate}
					\item 获取染色体长度。
						\begin{enumerate}
							\item 获取数据。在 \textcolor{purple}{UCSC Table} 界面中,修改 \textcolor{orange}{group} 为 \textcolor{cyan}{All Tables}、\textcolor{orange}{database} 为 \textcolor{cyan}{hg19}、\textcolor{orange}{table} 为 \textcolor{cyan}{chromInfo},待界面刷新后,直接点击 \textcolor{orange}{get output} 并继续点击 \textcolor{orange}{Send query to Galaxy} 即可将染色体的长度信息载入到工作空间中。
							\item 修改属性。为便于区分不同数据,可以点击 \textcolor{purple}{2:UCSC Main on Human: chromInfo (genome)} 右侧的铅笔图表,修改 \textcolor{orange}{Name} 为 \textcolor{cyan}{chromInfo},点击 \textcolor{orange}{Save} 更新数据属性。
						\end{enumerate}
				\end{enumerate}
			\item 计算基因数目。
				\begin{enumerate}
					\item
					  计算每条染色体上的基因数目。打开 \textcolor{purple}{Galaxy} 的 \textcolor{blue}{Statistics} 分组,点击其中的 \textcolor{blue}{Count} 工具。其中,\textcolor{orange}{from dataset} 选择 \textcolor{cyan}{1:refGene},\textcolor{orange}{Count occurrences of values in column(s)} 点选  \textcolor{cyan}{c1},最后点击 \textcolor{orange}{Execute} 即可计算出每条染色体上的基因数目。同样修改数据的属性,把 \textcolor{orange}{Name} 改为 \textcolor{cyan}{geneOnChromAll}。
					\item 过滤数据。打开 \textcolor{purple}{Galaxy} 的 \textcolor{blue}{Filter and Sort} 分组,点击其中的 \textcolor{blue}{Select} 工具。其中,\textcolor{orange}{Select lines from} 选择 \textcolor{cyan}{3:geneOnChromAll},\textcolor{orange}{that} 选择 \textcolor{cyan}{NOT Matching},\textcolor{orange}{the pattern} 填写 \textcolor{cyan}{\_},最后点击 \textcolor{orange}{Execute} 过滤数据。修改 \textcolor{orange}{Name} 为 \textcolor{cyan}{geneOnChromFilter}。
				\end{enumerate}
			\item 计算基因密度。
				\begin{enumerate}
					\item 整合染色体上的基因数目数据和染色体的长度信息。打开 \textcolor{purple}{Galaxy} 的 \textcolor{blue}{Join, Subtract and Group} 分组,点击其中的 \textcolor{blue}{Join two Datasets} 工具。其中,\textcolor{orange}{Join} 选择 \textcolor{cyan}{2:chromInfo},\textcolor{orange}{using column} 选择 \textcolor{cyan}{c1},\textcolor{orange}{with} 选择 \textcolor{cyan}{4: geneOnChromFilter},\textcolor{orange}{and column} 选择 \textcolor{cyan}{c2},其他参数默认即可,最后点击 \textcolor{orange}{Execute} 整合两套数据。
					\item 计算每条染色体上的基因密度。打开 \textcolor{purple}{Galaxy} 的 \textcolor{blue}{Text Manipulation} 分组,点击其中的 \textcolor{blue}{Compute} 工具。在 \textcolor{orange}{Add expression} 中填写 \textcolor{cyan}{c4/c2*100000000},其他参数默认即可,最后点击 \textcolor{orange}{Execute} 计算染色体上每100 Mb的基因数目。
					\item 提取染色体号、长度及其上的基因数目、基因密度等有用信息。打开 \textcolor{purple}{Galaxy} 的 \textcolor{blue}{Text Manipulation} 分组,点击其中的 \textcolor{blue}{Cut} 工具。在 \textcolor{orange}{Cut columns} 中填写 \textcolor{cyan}{c1,c2,c4,c6},其他参数默认即可,最后点击 \textcolor{orange}{Execute} 提取需要的几列。
					\item 根据染色体长度排序数据。打开 \textcolor{purple}{Galaxy} 的 \textcolor{blue}{Filter and Sort} 分组,点击其中的 \textcolor{blue}{Sort} 工具。其中,\textcolor{orange}{on column} 选择 \textcolor{cyan}{c2},其他参数默认,最后点击 \textcolor{orange}{Execute} 即可根据染色体长度将数据进行排序。修改 \textcolor{orange}{Name} 属性为 \textcolor{cyan}{geneNumberDensity}。
				\end{enumerate}
			\item 绘制并查看条形图。 
				\begin{enumerate}
					\item 绘图。打开 \textcolor{purple}{Galaxy} 的 \textcolor{blue}{Graph/Display Data} 分组,点击其中的 \textcolor{blue}{Bar chart} 工具。其中,在 \textcolor{orange}{Numerical columns} 中点选 \textcolor{cyan}{c3} 和 \textcolor{cyan}{c4},其他参数适当调整后,点击 \textcolor{orange}{Execute} 即可将上一步的结果可视化。
					\item 查看并保存图片。点击数据右侧的眼睛图表,可以在 \textcolor{purple}{Galaxy} 中查看绘制的条形图;也可点击数据左下方的软盘图表,将图片保存至本地。
					\item 尝试使用 \textcolor{blue}{Bar chart} 工具分开绘制基因数目和基因密度的条形图。
				\end{enumerate}
			\item 分析结果。如:哪条染色体上的基因数目最多/少?哪条染色体上的基因密度最高/低?数目最多和密度最高的染色体是不是同一条?从中学到了什么\footnote{提示:数据要经过标准化后才能互相比较。}?此外,还可以把得到的结果和教材或文献中的数据进行比较。
		\end{enumerate}
	\item[学时分配] \ 
		\begin{description}
			\item[设计讲解:] 2学时
			\item[实验:] 18学时
			\item[总结和课程设计报告:] 12学时 
			\item[答辩:] 4学时
			\item[总共:] 36学时 
			\item[地点:] 教一楼205室生物信息学实验室 
		\end{description}
	\item[资源网站] \ 
		\begin{enumerate}
			\item Galaxy Test:\href{https://test.g2.bx.psu.edu/}{https://test.g2.bx.psu.edu/}
			\item Galaxy Main:\href{https://main.g2.bx.psu.edu/}{https://main.g2.bx.psu.edu/}
		\end{enumerate}
\end{description}


\newpage
%Galaxy2
\begin{description}
	\item[名称:] 基于Galaxy平台分析***物种SNP在不同特征区域中的分布
	\item[总学时:] 36
	\item[学分:] 2
	\item[对象:] 生物医学工程系生物技术与生物信息专业本科生
	\item[指导教师:] 伊现富
	\item[课程设计目的:] 在学习分子生物学、生物计算技术、生物网络数据库、生物信息学课程的基础上,培养学生分析大规模基因组数据的能力。此题目旨在帮助学生熟悉易用、强大且日渐流行的Galaxy生物信息学分析平台,并使用此平台处理基因组数据,分析SNP在不同特征区域上的分布情况。
	\item[实验内容] \
		\begin{enumerate}
			\item 选择合适的物种\textit{S}。
			\item 熟悉Galaxy界面。
			\item 获取该物种不同特征区域(如:5' UTR外显子、编码区外显子和3' UTR外显子)的信息。
			\item 获取该物种dbSNP数据库的SNP信息。
			\item 计算不同特征区域的SNP数目和密度。
			\item 将SNP密度进行标准化。
			\item 比较不同特征区域的SNP密度。
			\item 对所得结果进行综合分析。
		\end{enumerate}
	\item[技术指标] \
		\begin{enumerate}
			\item 基因组范围不同特征区域的数据。
			\item 基因组范围上SNP的数据。
			\item 不同特征区域SNP的数目。
			\item 不同特征区域SNP的密度。
			\item 标准化之后的SNP密度。
			\item 对所得结果的分析与理解。
			\item 课程设计报告书:(1)名称;(2)目的和任务;(3)实验步骤;(4)实验结果;(5)实验结果分析和讨论。
			\item 课程设计答辩的PPT文件(答辩时间5分钟)。
		\end{enumerate}
	\item[实验步骤] \ 
		\begin{enumerate}
			\item 选择完成基因组测序且注释比较完善的物种,如:人、小鼠,等。\footnote{此处以“人”为例详述步骤。}
			\item 打开 \textcolor{purple}{Galaxy Main} 主页,熟悉其界面布局。
			\item 获取基因组范围不同特征区域的基本信息。
				\begin{enumerate}
					\item 获取数据。打开 \textcolor{purple}{Galaxy} 的 \textcolor{blue}{Get Data} 分组,点击\textcolor{blue}{UCSC Main} 进入 \textcolor{purple}{UCSC Table} 界面。在 \textcolor{purple}{UCSC Table} 界面中,\textcolor{orange}{clade} 选择 \textcolor{cyan}{Mammal}、\textcolor{orange}{genome} 选择 \textcolor{cyan}{Human}、\textcolor{orange}{assembly} 选择 \textcolor{cyan}{Feb. 2009 (GRCh37/hg19)}、\textcolor{orange}{group} 选择 \textcolor{cyan}{Genes and Gene Prediction Tracks}、\textcolor{orange}{track} 选择 \textcolor{cyan}{RefSeq Genes}、\textcolor{orange}{table} 选择 \textcolor{cyan}{refGene}、\textcolor{orange}{region} 点选 \textcolor{cyan}{genome}、\textcolor{orange}{output format} 选择 \textcolor{cyan}{BED – browser extensible data}、\textcolor{orange}{Send output to} 勾选 \textcolor{cyan}{Galaxy}、\textcolor{orange}{file type returned} 点选 \textcolor{cyan}{plain text},之后点击 \textcolor{orange}{get output},新界面中的 \textcolor{orange}{Create one BED record per} 点选 \textcolor{cyan}{5' UTR Exons},最后点击 \textcolor{orange}{Send query to Galaxy} 即可将5' UTR外显子的基本信息载入到 \textcolor{purple}{Galaxy} 平台的工作空间中。
					\item 修改属性。为了便于区分工作空间中的不同数据,可以点击 \textcolor{purple}{UCSC Main on Human: refGene (genome)} 数据集右侧的铅笔图标,修改数据集的属性,如:修改 \textcolor{orange}{Name} 为 \textcolor{cyan}{5UTR},之后点击 \textcolor{orange}{Save} 更新其属性。
					\item 重复操作。重复上述步骤,在 \textcolor{orange}{Create one BED record per} 分别点选 \textcolor{cyan}{Coding Exons} 和 \textcolor{cyan}{3' UTR Exons},将编码区外显子和3' UTR外显子的基本信息载入到 \textcolor{purple}{Galaxy} 平台的工作空间中。
				\end{enumerate}
			\item 获取dbSNP数据库中SNP的信息。
				\begin{enumerate}
					\item 获取数据。在 \textcolor{purple}{UCSC Table} 界面中,修改 \textcolor{orange}{group} 为 \textcolor{cyan}{Variation and Repeats}、\textcolor{orange}{track} 为 \textcolor{cyan}{Common SNPs(135)}、\textcolor{orange}{table} 为 \textcolor{cyan}{snp135Common},待界面刷新后,点选 \textcolor{orange}{Create one BED record per} 中的 \textcolor{cyan}{Whole Gene},最后点击 \textcolor{orange}{Send query to Galaxy} 即可将SNP的基本信息载入到 \textcolor{purple}{Galaxy} 平台的工作空间中。
					\item 修改属性。为便于区分不同数据,可以点击 \textcolor{purple}{UCSC Main on Human: snp135Common (genome)} 数据集右侧的铅笔图表,修改 \textcolor{orange}{Name} 为 \textcolor{cyan}{SNP135},点击 \textcolor{orange}{Save} 更新数据属性。
				\end{enumerate}
			\item 计算不同特征区域中的SNP数目及其密度。 
				\begin{enumerate}
					\item 关联特征区域和SNP的信息。打开 \textcolor{purple}{Galaxy} 的 \textcolor{blue}{Operate on Genomic Intervals} 分组,点击其中的 \textcolor{blue}{Join} 工具。其中,\textcolor{orange}{Join(First dataset)} 选择 \textcolor{cyan}{5UTR} 数据,\textcolor{orange}{with(Second dataset)} 选择 \textcolor{cyan}{SNP135} 数据,其余参数默认即可,最后点击 \textcolor{orange}{Execute} 即可将特征区域的信息与SNP的信息关联起来。\footnote{注意:此步舍弃了没有SNP的特征区域。}
					\item 计算特征区域中的SNP数目。打开 \textcolor{purple}{Galaxy} 的 \textcolor{blue}{Statistics} 分组,点击其中的 \textcolor{blue}{Count} 工具。其中,\textcolor{orange}{from dataset} 选择上一步的结果,\textcolor{orange}{Count occurrences of values in column(s)} 点选 \textcolor{cyan}{c4},最后点击 \textcolor{orange}{Execute} 即可计算出每个5' UTR外显子上的SNP数目。
					\item 恢复计数结果中的特征区域信息。打开 \textcolor{purple}{Galaxy} 的 \textcolor{blue}{Join, Subtract and Group} 分组,点击其中的 \textcolor{blue}{Join two Datasets} 工具。其中,\textcolor{orange}{Join} 选择上一步的结果,\textcolor{orange}{using column} 选择 \textcolor{cyan}{c2},\textcolor{orange}{with} 选择 \textcolor{cyan}{5UTR} 数据,\textcolor{orange}{and column} 选择 \textcolor{cyan}{c4},其他参数默认即可,最后点击 \textcolor{orange}{Execute} 整合两套数据。
					\item 计算每个特征区域的长度。打开 \textcolor{purple}{Galaxy} 的 \textcolor{blue}{Text Manipulation} 分组,点击其中的 \textcolor{blue}{Compute} 工具。在\textcolor{orange}{Add expression} 中填写 \textcolor{cyan}{c5-c4},其他参数默认即可,最后点击 \textcolor{orange}{Execute} 计算每个特征区域的长度。
					\item 提取数据集中的有用信息。打开 \textcolor{purple}{Galaxy} 的 \textcolor{blue}{Text Manipulation} 分组,点击其中的 \textcolor{blue}{Cut} 工具。在 \textcolor{orange}{Cut columns} 中填写 \textcolor{cyan}{c3,c9,c1},其他参数默认即可,最后点击 \textcolor{orange}{Execute} 提取需要的几列。
					\item 按染色体整理数据。打开 \textcolor{purple}{Galaxy} 的 \textcolor{blue}{Join, Subtract and Group} 分组,点击其中的 \textcolor{blue}{group} 工具。其中,\textcolor{orange}{Select data} 选择上一步的结果,\textcolor{orange}{Group by column} 选择 \textcolor{cyan}{c1};之后,点击 \textcolor{orange}{Add new Operation},\textcolor{orange}{Type} 选择 \textcolor{cyan}{Sum},\textcolor{orange}{on column} 选择 \textcolor{cyan}{c2};之后,继续点击 \textcolor{orange}{Add new Operation},\textcolor{orange}{Type} 选择 \textcolor{cyan}{Sum},\textcolor{orange}{on column} 选择 \textcolor{cyan}{c3};其他参数默认,最后点击\textcolor{orange}{Execute} 即可。
					\item 过滤数据。打开 \textcolor{purple}{Galaxy} 的 \textcolor{blue}{Filter and Sort} 分组,点击其中的 \textcolor{blue}{Select} 工具。其中,\textcolor{orange}{Select lines from} 选择上一步的结果,\textcolor{orange}{that} 选择 \textcolor{cyan}{NOT Matching},\textcolor{orange}{the pattern} 填写  \textcolor{cyan}{\_},最后点击 \textcolor{orange}{Execute} 过滤数据。 
					\item 计算每个特征区域上SNP的密度。打开 \textcolor{purple}{Galaxy} 的 \textcolor{blue}{Text Manipulation} 分组,点击其中的 \textcolor{blue}{Compute} 工具。在 \textcolor{orange}{Add expression} 中填写 \textcolor{cyan}{c3/c2*1000},其他参数默认即可,最后点击 \textcolor{orange}{Execute} 计算SNP的密度。将 \textcolor{orange}{Name} 属性修改为 \textcolor{cyan}{5UTRSNP}。
				\end{enumerate}
			\item 标准化SNP密度。
				\begin{enumerate}
					\item 计算基因组上特征的总长度及SNP的总数目。打开 \textcolor{purple}{Galaxy} 的 \textcolor{blue}{Statistics} 分组,点击其中的 \textcolor{blue}{Summary Statistics} 工具。其中,\textcolor{orange}{Summary statistics on} 选择 \textcolor{cyan}{5UTRSNP},\textcolor{orange}{Column or expression} 填写 \textcolor{cyan}{c2} 或 \textcolor{cyan}{c3},最后点击 \textcolor{orange}{Execute} 分别计算特征总长L及SNP的总数N。
					\item 将每条染色体上的SNP密度进行标准化。首先计算 $M=N/L*1000$ 。然后,打开 \textcolor{purple}{Galaxy} 的 \textcolor{blue}{Text Manipulation} 分组,点击其中的 \textcolor{blue}{Compute} 工具。在 \textcolor{orange}{Add expression} 中填写 \textcolor{cyan}{c4-M},\textcolor{orange}{as a new column to} 选择 \textcolor{cyan}{5UTRSNP},最后点击 \textcolor{orange}{Execute} 标准化SNP密度。将 \textcolor{orange}{Name} 属性修改为 \textcolor{cyan}{5UTRSNPNormalized}。
				\end{enumerate}
			\item 比较不同特征区域的SNP密度。
				\begin{enumerate}
					\item 重复第5、6两步\footnote{提示:可以提取工作流程后,修改输入数据,自动化重复前述操作!},获得其他特征区域的SNP密度。
					\item 比较不同特征区域的SNP密度。打开 \textcolor{purple}{Galaxy} 的 \textcolor{blue}{Join, Subtract and Group} 分组,点击其中的 \textcolor{blue}{Join two Datasets} 工具。其中,\textcolor{orange}{Join} 选择 \textcolor{cyan}{5UTRSNPNormalized} 数据,\textcolor{orange}{using column} 选择 \textcolor{cyan}{c1},\textcolor{orange}{with} 选择 \textcolor{cyan}{codingSNPNormalized} 数据,\textcolor{orange}{and column} 选择 \textcolor{cyan}{c1},其他参数默认即可,最后点击 \textcolor{orange}{Execute} 整合5' UTR和编码区两套数据。以类似的操作,再将3' UTR数据整合进去。
				\end{enumerate}
			\item 综合分析所得结果\footnote{\textcolor{red}{敬请注意:}因为处理过程中的取舍问题,请不要把此处的计算结果当作常识来使用!}。如:对于同一条染色体来说,不同特征区域的SNP密度有没有差别?对于同一类特征区域来说,不同染色体的SNP密度有没有差别?此外,还可以将两者结合起来进行分析,并尝试对分析结果进行解释。
		\end{enumerate}
	\item[学时分配] \ 
		\begin{description}
			\item[设计讲解:] 2学时
			\item[实验:] 18学时
			\item[总结和课程设计报告:] 12学时 
			\item[答辩:] 4学时
			\item[总共:] 36学时 
			\item[地点:] 教一楼205室生物信息学实验室 
		\end{description}
	\item[资源网站] \ 
		\begin{enumerate}
			\item Galaxy Main:\href{https://main.g2.bx.psu.edu/}{https://main.g2.bx.psu.edu/}
			\item Galaxy Test:\href{https://test.g2.bx.psu.edu/}{https://test.g2.bx.psu.edu/}
		\end{enumerate}
\end{description}
\end{document}
